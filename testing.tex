	%*****************************************************************************
	\chapter{Testing}	
		
	The goal was evaluate the usability of interface and usability of proposed methods for lacalizing blind pedestrians. 
	
	method - Low-fi 1
	(jak jsme to navrhli (jak jsme to navrhli a jak jsme implementovali vstupy z Low-fi 1))
	Evaluation
	-participants (vek, pocet, odkud, jaci lide)
	-procedure (jak jsem to delal (wizard-of-ozz, apod, co bylo za ukol, co jim experimentator rekl, instrukce)
	-apparatus (technicky setup, technologie, jak se to zaznamenalo)
	-design* (skupiny, shuffling, nezavisle, zavisle promenne)
	-results\&discussion
	-recommendation for design
	
	\section{Contacts pool} \label{sec:contactsPool}
	I was collecting the contacts of blind people using the Snowball method. The seeds were Invisible exhibition, their personal websites, friends of my friends and blinds in streets.
	
	I visited the exhibition in Prague called Invisible exhibition \cite{later} and gathered few phone numbers there. I hit into personal websites of some blind by googling for topics around blindness. (I am not making citations to their websites, with respect to their privacy). My friends as well knew some blind in their surroundings and provided me with a connection to them and last I actively started to ask blind, when I saw them with the white cone on the street. All these people willingly provided me with contacts on their friends, which would be interested in trying the technology for themselves so the snowball effect continued.
	\section{Participants}
	
	5 blind people participated in the study. I sent emails and sms based on my contact library (see \ref{sec:contactsPool}) and the first 5 to be available soonest participated in the study. 
	
	The age of participants was ranging from 28 to 68.
	They were totally blind or just with a point vision (imagine looking through the paper with a small gap made by a needle).
	All of them were blind since childhood and one of them was using dog.
	\section{Procedure}
	(jak jsme to navrhli)
	Evaluation
	-participants (vek, pocet, odkud, jaci lide)
	-procedure (jak jsem to delal (wizard-of-ozz, apod, co bylo za ukol, co jim experimentator rekl, instrukce)
	-apparatus (technicky setup, technologie, jak se to zaznamenalo)
	-design* (skupiny, shuffling, nezavisle, zavisle promenne)
	-results\&discussion
	-recommendation for design
	
	\subsection{Setup}
	We were testing 3 prototypes: \poi{}, \reversegeo{} and \gps{}. The prototypes are deeply described in section \ref{sec:prototypes}. 
	\subsection{}
	\section{Apparatus}
	\subsection{Technology}
	All the prototypes were interactive websites and running in a Chrome browser on the cellphone Huawei Honor 7 lite, 5" inches screen with Android 7 Nougat. The websites were read aloud and controlled through screen reader Google TalkBack\cite{later}.
	\subsection{data collection}
	The sessions were
	
	\section{Design}
	\subsection{Locations}
	Initial places depended on the prototype:
	\begin{description}
		\item [\poi{}] randomly choosen tramstop, and we arrived there by tram
		\item [\reversegeo{}] randomly choosen place, while walking through the city
		\item [\gps{}] randomly choosen place, while walking through the city
	\end{description}
	
	The initial place never repeated within more participants. It could repeat within one participant.
	\subsection{Method orders}
	\begin{table}[]
		\centering
		\caption{Order of procedures}
		\label{my-label}
		\begin{tabular}{@{}cccc@{}}
			\toprule
			\textbf{participant id} & \textbf{1st} & \textbf{2nd} & \textbf{3rd} \\ \midrule
			\#1                     & POI          & RG           & GPS          \\
			\#2                     & POI          & RG           & GPS          \\
			\#3                     & RG           & GPS          & POI          \\
			\#4                     & GPS          & POI          & RG           \\
			\#5                     & RG           & GPS          & POI          \\ \bottomrule
		\end{tabular}
	\end{table}
	
	(skupiny, shuffling, nezavisle, zavisle promenne)
	\section{Results \& discussion}
	\section{Recommendation for design}